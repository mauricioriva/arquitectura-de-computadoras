\documentclass[14pt,letterpaper,fleqn]{article}

\usepackage[utf8]{inputenc}
\usepackage[spanish,es-nodecimaldot]{babel}
\usepackage{amsmath}
\usepackage{amssymb}
\usepackage{multicol}
\usepackage{graphicx}
\usepackage{cancel}

\usepackage[dvipsnames]{xcolor}
\usepackage[most]{tcolorbox}

\usepackage{tabu}

\usepackage{pgfplots}
\pgfplotsset{width=10cm,compat=1.9}

\usepackage{mathtools}
\usepackage{tikz}
\usetikzlibrary{trees,positioning}

\usepackage[top=1in, bottom=1in, left=1in, right=1in]{geometry}


\begin{document}

\begin{titlepage}
    \centering

    {\scshape\LARGE Universidad Nacional Autónoma de México \par}

    \vspace{1cm}
    {\scshape\Large Facultad de ciencias\par}
    \vspace{1.5cm}

    \begin{center}
        \includegraphics[scale=.1]{../../assets/img/logo.png}
    \end{center}

    \vspace{.8 cm}

    {\LARGE Tarea  03 : \par}
    {\huge\bfseries Organización y Arquitectura de Computadoras \par}

    \vspace{0.5cm}
    \large{\itshape{Martin Felipe Espinal Cruces }} \small{ - 316155362 } \\ \vspace{0.3cm}
   
    \vfill

    Trabajo presentado como parte del curso de
    \textbf{Organizacion y Arquitectura}
    impartido por el profesor \textbf{	José de Jesús Galaviz Casas }. \par
    \vspace{0.1cm}
\end{titlepage}

    \begin{enumerate}
    
        %Ejercicio 1%
        \item \textbf{¿Cuáll es el esquema general de la arquitectura de Von Neumann?}
        \begin{center}
                \includegraphics[scale=.7]{../../assets/img/neumann.jpg}
            \end{center}
		%Ejercicio 2
        \item \textbf{¿Cuáles los niveles en los cuales se clasifican los lenguales de programación, y menciona un ejemplo de cada uno.}
        \\Los leguajes de programación se clasifican en tres niveles, alto, medio bajo nivel.\\
        En el bajo nivel, el lenguaje que se utiliza para realizar las instrucciones solicitadas se encuentran en binario; un ejemplo de lenguaje de programación es ensamblador.\\
        El en nivel medio, el lenguaje es un poco más abstracto que en el nivel anterior, sin embargo se preservan ciertas caracteristícas del lenguaje de máquina como ciertas instrucciones; un ejemplo de lenguaje de lenguaje de programación en este nivel es C.\\
        Finalmente tenemos el nivel alto, en donde el lenguaje utilizado para realizar las instrucciones necesarias se encuentran en una sintáxis más familiar a nuestra forma de comunicarnos; un ejemplo de lenguaje de programación en este nivel es Python.
        
        %Ejercicio 3%
        \item \textbf{Supon que el método M cuenta con el 30\% del tiempo de la ejecución de un programa. Sea $s'_n$ el sppedup con n procesos. Tu jefe te dice que debes duplicar este speedup: la versión nueva del programa debe tener un speedup $s_n \geq  2 * s'_n$. Tu buscas a un programador para remplazar M con una versión mejorada, k veces más rápida. ¿Qué valor de k es requerido? (Hint: para sacar el speedup usa la ley de Amdalh)}
        \\Para comenzar debemos sacar la ganancia neta la cual se calcula de la siguiente manera:

            $$g = \frac{Tiempo_{sin}}{Tiempo_{con}} \Rightarrow{}
             g = \frac{70\%}{30\%}  = 2.33$$
        Ahora sustituimos en la ley de Amdalh obteniendo lo siguiente
        \begin{equation*}
            \frac{1}{(1-.3)+\frac{.3}{2.33}} = \frac{1}{.7+\frac{.3}{2.3}}
            =  \frac{1}{.7+1.2857} = \frac{1}{.8285} =1.2070
        \end{equation*}
        Dado que queremos mejorar la ganancia bruta k veces de tal manera que obtengamos un speedup $\geq 2 * s'_n$, tendremos que:
            \begin{equation}
                \begin{split}
                \frac{1}{(1-F)+\frac{F}{2.33}} & = (1.2070) * 2\\
                 & = 2.4136 \\
                 (1-F) + \frac{F}{2.33} & = \frac{1}{2.414} \\
                 1 - F + 0.4291F & = 0.4142 \\
                 0.4291F - F & = 0.4142 - 1 \\
                 -0.5709F & = -0.5857\\
                 F & = \frac{-0.5857}{-0.5709}\\
                 & = 1.0260
                \end{split}
            \end{equation}
        Dado que el speedup resultante es el requerido sólo basta ver que:
        \begin{equation*}
            \begin{split}
                F*k & = 1.0260\\
                (.3) * k & = 1.0260\\
                k & = \frac{1.0260}{.3}\\
                k & = 3.42
            \end{split}{}
        \end{equation*}{}
        Por lo tanto el valor de k requerido es de 3.42
        %Ejercicio 4%
        \item \textbf{Supón que el método M se puede acelerar tres veces. ¿Qué fracción de todo el tiempo de ejecución debe contar M para que se pueda doblar el speedup del programa?}
        \\El problema nos indica que debemos encontrar el valor de la ganancia bruta cuando g = 3 y el speedup el = 2*(1.0259), entonces:
            \begin{equation*}
                \begin{split}
                \frac{1}{(1-F)+\frac{F}{3}} & = (1.0260) * 2\\
                 & = 2.4136 \\
                 (1-F) + \frac{F}{3} & = \frac{1}{2.052} \\
                 1 - F + \frac{F}{3} & = 0.4873 \\
                 -\frac{2F}{3} & = 0.4873 - 1 \\
                  & = \\
                 -2F & = \frac{−0.5126}{3}\\
                 & = −1.5378\\
                 F &= 0.7689
                \end{split}
            \end{equation*}
        %Ejercicio 5%
        \item \textbf{De las siguientes expresiones, minimízalas con álgebra booleana y comprueba tu resultado usando mapa de Karanauht}
        \begin{enumerate}
            %Ejercicio 5.a%
            \item \textbf{$F(x_0, x_1, x_2) = \bar{x_0}\bar{x_1}\bar{x_2} + \bar{x_0}\bar{x_1}x_2 + x_0\bar{x_1}\bar{x_2} + \bar{x_0}x_1x_2 + \bar{x_0}\bar{x_1}x_2 + x_0x1_x_2$}
            \\
            
            %Ejercicio 5.b%
            \item \textbf{$F(x_0, x_1, x_2, x_3) = \bar{x_0}\bar{x_1}\bar{x_2}\bar{x_3} + \bar{x_0}\bar{x_1}x_2\bar{x_3} + x_0\bar{x_1}\bar{x_2}\bar{x_3} + \bar{x_0}x_1x_2\bar{x_3} + \bar{x_0}x_1\bar{x_2}\bar{x_3} + \bar{x_0}\bar{x_1}\bar{x_2}x_3 + \bar{x_0}\bar{x_1}x_2x_3 + x_0\bar{x_1}\bar{x_2}x_3 + \bar{x_0}x_1x_2x_3 + x_0x_1x_2x_3$}
            
        \end{enumerate}
        
    \end{enumerate}
\end{document}